\documentclass[11pt,psfig]{article}
\usepackage{epsfig}
\usepackage{times}
\usepackage{amssymb}
\usepackage{float}

\newcount\refno\refno=1
\def\ref{\the\refno \global\advance\refno by 1}
\def\ux{\underline{x}}
\def\ut{\underline{\theta}}
\def\umu{\underline{\mu}}
\def\be{p_e^*}
\newcount\eqnumber\eqnumber=1
\def\eq{\the \eqnumber \global\advance\eqnumber by 1}
\def\eqs{\eq}
\def\eqn{\eqno(\eq)}

 \pagestyle{empty}
\def\baselinestretch{1.1}
\topmargin1in \headsep0.3in
\topmargin0in \oddsidemargin0in \textwidth6.5in \textheight8.5in
\begin{document}
\setlength{\parskip}{1.2ex plus0.3ex minus 0.3ex}


\thispagestyle{empty} \pagestyle{myheadings} \markright{Homework
5: CS 274A, Probabilistic Learning: Winter 2014}



\title{CS 274A Homework 5}
\author{Zachary DeStefano, 15247592}
\date{Due Date: Wednesday March 5th}

\maketitle

 \newpage

\section*{What to Submit}

 
\begin{itemize} \item Submit your MATLAB  code to the
Homework 5 dropbox in EEE, including a script that loads each data set and runs your code on each data set. Please upload all of your functions in a single compressed
file (please use .gz or zip format). If you are using something other than MATLAB please also provide a README file to explain to us how to use your code.

\item Submit a written hardcopy summary of your results in class with the information below.
\end{itemize}


 Please try to put multiple plots on the same page using ``subplots" in MATLAB
(where different pages could correspond to different data sets and different
algorithms), to avoid printing lots of pages. For each data set, using the
true $K$ value for each one, show the following
\begin{enumerate}
\item The $K$-means solution (scatter plot in
 two dimensions (any two dimensions for Data Set 4)
 illustrating the location of the solution (i.e.,
 the cluster means), and plotting the data from different
 clusters with different symbols (and/or in color if you would like to use color).
\item A plot of the sum-squared-error (divided by $n$) as a function of
iteration number in the $K$-means algorithm. 
\item The initial parameter values
and the final parameter values (2 plots, each showing means and covariances for
each cluster) for the EM/Gaussian mixtures code  for the
highest-likelihood solution. You can use or modify the code from Homework 1 to do the plotting. 

\item A plot of the log-likelihood (for one run of your algorithm) as a
function of iteration number during EM. 

\item Add some brief comments (1
paragraph) on the difference between $K$-means and EM for each data set. 
\item  Generate a table of log-likelihood and BIC scores for
$K$ going from $K=1$ to some maximum value (e.g., $K=5$). Comment briefly on the results.  
%\item For data set 4, generate an image where you replace each original pixel value with the mean of the cluster it is assigned to by $K$-means, for $K=5, 10, 15$. This is essentially how image data compression works. Comment on the visual quality of the ``compressed" images (with the pixel rgb values replaced by their cluster means) compared to the original images.

%\item An additional 2 to 3 page report for the extra credit portion, if you do the extra credit part. Your extra credit report should be clearly labeled with "ExtraCredit" in the title.

\end{enumerate}




\end{document}
